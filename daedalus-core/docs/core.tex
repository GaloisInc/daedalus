\documentclass{article}

\usepackage{amsmath}
\usepackage{amssymb}
\usepackage{stmaryrd}
\usepackage{mathpartir}
\usepackage{stmaryrd}
\usepackage{float}


\newcommand{\sem}[1]{\llbracket{#1}\rrbracket}
\newcommand{\setSem}[1]{{\sem {#1}}^{\mathrm{set}}}
\newcommand{\funSem}[1]{{\sem {#1}}^{\mathrm{fun}}}
\newcommand{\fun}[1]{\mathrm{#1}~}
\newcommand{\set}[1]{\{{#1}\}}
\newcommand{\prefixOf}[2]{{#1}~\mathrm{prefixOf}~{#2}}
\newcommand{\caseSelect}[2]{\fun{select}{#1}~{#2}}

\newcommand{\env}[2]{{#1}\vdash{#2}}
\newcommand{\expr}[2]{{#1}~\rightarrow~{#2}}
\newcommand{\parser}[4]{{#1}~\rightarrow~{#2}\rhd{#3}\cdot{#4}}
\newcommand{\rejectsPrefixesOf}[2]{{#2}\notin{#1}}


\newcommand{\eString}[1]{\mathtt{"{#1}"}}

\newcommand{\gPure}[1]{\mathbf{pure}~{#1}}
\newcommand{\gFail}{\mathbf{fail}}
\newcommand{\gMatch}[1]{\mathbf{match}~{#1}}
\newcommand{\gGet}[1]{\mathbf{get}{#1}}
\newcommand{\gBind}[3]{{#1} = {#2}; {#3}}
\newcommand{\gOr}[2]{{#1} \talloblong {#2}}
\newcommand{\gOrB}[2]{{#1} \triangleleft {#2}}
\newcommand{\gParse}[2]{\mathbf{parse}~{#1}~{#2}}
\newcommand{\gGetS}{\mathbf{peek}}
\newcommand{\gSetS}[1]{\mathbf{setStream}~{#1}}
\newcommand{\gEnd}{\mathbf{end}}
\newcommand{\gCase}[2]{\mathbf{case}~{#1}~\mathbf{of}~{#2}}
\newcommand{\gIf}[3]{\mathbf{if}~{#1}~\mathbf{then}~{#2}~\mathbf{else}~{#3}}
\newcommand{\gCall}[2]{{#1}~{#2}}

\newcommand{\from}{\leftarrow}
\newcommand{\append}{+\!\!\!+}
\newcommand{\conj}{\wedge}
\newcommand{\union}{\cup}
\newcommand{\nat}{\mathbb{N}}




\begin{document}

\section{Syntax}

\begin{figure}[H]
{\setlength{\arraycolsep}{0pt}
\[
\begin{array}{rcl}
S &{}={}& \gPure E \\
  &{}|{}& \gBind x {S_1} {S_2} \\
  &{}|{}& \gCall f {E^*} \\
\\
  &{}|{}& \gFail \\
  &{}|{}& \gOr {S_1} {S_2} \\
  &{}|{}& \gOrB {S_1} {S_2} \\
\\
  &{}|{}& \gGet E \\
  &{}|{}& \gGetS \\
\\
  &{}|{}& \gParse S E \\
  &{}|{}& \gCase E {(P \to S)^+} \\
\\
E &{}={}& \text{\dots language of expressions\dots} \\
\end{array}
\]
}
\caption{The Core language}
\end{figure}


\section{Semantics}

In this section we present a few different formulations of the semantics
of the Core language.  Throughout, we use the following notation:
\[
\begin{array}{ll}
I       & \text{The type of streams of tokens (i.e., strings)} \\
X,Y,Z   & \text{Refer to elements of $I$} \\
\append & \text{Concatenates members of $I$} \\
V       & \text{The type of semantic values} \\
u,v     & \text{Refer to elements of $V$}
\end{array}
\]

The dynamic environments are implicit, except in the relational specification.
The notation $\sem{ \_ }^{x=v}$ indicates the dynamic environment is extended
with a new binding of variable $x$ to $v$.

We won't specify the details of expression language is as it is
somewhat orthogonal to the semantics of parsers:
\[
\sem E : V
\]


\subsection{Semantics as a State Transformer}

One way to give semantics is to model them as a state transformer:
\[
\sem S : I \to \set{ (V,I) }
\]

\begin{figure}[H]
\begin{align*}
\sem {\gPure E}~X &= \set{ (\sem E, X) } \\
\sem {\gBind x {S_1} {S_2}}~X &=  \set{ (v,Z) ~|~ (u,Y) \from \sem {S_1}~X,
                                        (v,Z) \from \sem {S_2}^{x=u}~Y } \\
\sem {\gFail}~X   &= \set{ } \\
\sem {\gOr {S_1} {S_2}}~X &= \sem{S_1}~X \union \sem{S_2}~X \\
\sem {\gOrB {S_1} {S_2}}~X &=
  \begin{cases}
  \sem {S_1}~X &\text{if}~\sem{S_1}~X \neq \emptyset \\
  \sem {S_2}~X &\text{otherwise}
  \end{cases} \\
\sem {\gGet E}~X &=
  \begin{cases}
  \set{ (Y,Z) } &\text{if}~|Y| = \sem{E} \conj X = Y \append Z \\
  \emptyset     &\text{otherwise}
  \end{cases} \\
\sem {\gGetS}~X &= \set {(X,X)} \\
\sem {\gParse S E}~X &= \set{ (v,X) ~|~ (v,Y) \from \sem S~\sem E } \\
\sem {\gCase E A}~X &= \sem {\caseSelect {\sem E} A}~X
\end{align*}
\caption{State transformer semantics of Core parsers.}
\end{figure}




\subsection{Set Semantics}

An alternative to state transformers is to describe a parser is a set
of triples:
\[
\sem S : \set{ (V,I,I) }
\]

If $(v,X,Y)$ is in the semantics of $S$, then when applied to
input $X \append Y$, $S$ will consume $X$ and produce result $v$.
This formulation allows us to talk about parsers in context.
A parser {\em accepts} an input if it doesn't fail on it:
\begin{align*}
\fun{accepts} S~(X \append Y) &= \exists v. (v,X,Y) \in \sem S \\
\end{align*}

\begin{figure}[H]
\begin{align*}
\sem {\gPure E} &= \set{ (\sem E, [], X) } \\
\sem {\gBind x {S_1} {S_2}} &=
  \set{ (v,X \append Y,Z) ~|~ (u,X,Y \append Z) \from \sem{S_1},
                              (v,Y,Z)           \from \sem{S_2}^{x=u} } \\
\sem{\gFail} &= \set{ } \\
\sem{\gOr {S_1} {S_2}} &= \sem{S_1} \union \sem{S_2} \\
\sem{\gOrB {S_1} {S_2}} &= \sem{S_1} \union
  \set { (v,X,Y) ~|~ (v,X,Y) \from \sem{S_2},
                     \fun{not} (\fun{accepts} S_1~(X \append Y)) } \\
\sem{\gGet E} &= \set{ (X, X, Y) ~|~ |X| = \sem{E} } \\
\sem{\gGetS}  &= \set{ (X, [], X) } \\
\sem{\gParse S E} &= \set{ (v,[],Z) ~|~ (v,X,Y) \from \sem{S},
                                        \sem{E} = X \append Y} \\
\sem{\gCase E A} &= \sem {\caseSelect {\sem E} A}
\end{align*}
\caption{Set semantics of Core parsers.}
\end{figure}

Example of a parser that depends on context:
\begin{equation*}
S = \gBind x \gGetS {\gCase x {\set{[] \to \gFail; \_ \to \gPure {()}}}}
\end{equation*}
This parser accepts the empty string, but only if is not at the end
of the input.


\subsection{Semantics as a Relation}

This is an alternative presentation of the set semantics.

\begin{figure}[H]
\begin{mathpar}

\inferrule[Pure]
  {\env \Gamma {\expr E v}}
  {\env \Gamma {\parser {\gPure E} v {[]} X}}

\inferrule[Advance]
  {\env \Gamma {\expr E |X|}}
  {\env \Gamma {\parser {\gGet E} X X Y}}

\inferrule[Look-Ahead]
  { }
  {\env \Gamma {\parser \gGetS X {[]} X}}

\inferrule[Sequnce]
  { \env \Gamma           {\parser {S_1} u X {Y \append Z}} \\
    \env {\Gamma, x = u } {\parser {S_2} v Y Z}
  }
  {\env \Gamma {\parser {\gBind x {S_1} {S_2}} v {X \append Y} Z}}

\inferrule[Unbiased-Choice-Left]
  {\env \Gamma {\parser {S_1} v X Y}}
  {\env \Gamma {\parser {\gOr {S_1} {S_2}} v X Y}}

\inferrule[Unbiased-Choice-Right]
  {\env \Gamma {\parser {S_2} v X Y}}
  {\env \Gamma {\parser {\gOr {S_1} {S_2}} v X Y}}

\inferrule[Biased-Choice-Left]
  {\env \Gamma {\parser {S_1} v X Y}}
  {\env \Gamma {\parser {\gOrB {S_1} {S_2}} v X Y}}

\inferrule[Biased-Choice-Right]
  {\env \Gamma {\parser {S_2} v X Y} \\
   \env \Gamma {\rejectsPrefixesOf {S_1} {(X \append Y)}}
  }
  {\env \Gamma {\parser {\gOrB {S_1} {S_2}} v X Y}}

\inferrule[Nested-Parser]
  { \env \Gamma {\expr E {X \append Y}} \\
    \env \Gamma {\parser S v X Y} }
  {\env \Gamma {\parser {\gParse S E} v {[]} Z}}

\inferrule[Case]
  {\env \Gamma {\expr E u} \\
   \env \Gamma {\parser {\caseSelect u A} v X Y}
  }
  {\env \Gamma {\parser {\gCase E A} v X Y}}

\end{mathpar}
\caption {
$\env \Gamma {\parser S v X Y}$ describes the behavior or parser $S$
in dynamic environment $\Gamma$.  When applied to the input $X \append Y$, $S$ will consume $X$ and produce semantic value $v$.}
\end{figure}


\begin{figure}[H]
\begin{mathpar}

\inferrule[Empty]
  { }
  {\env \Gamma {\rejectsPrefixesOf {\gFail} X}}

\inferrule[Too-Short]
  { \env \Gamma {\expr E v} \\
    |X| < v
  }
  {\env \Gamma {\rejectsPrefixesOf {\gGet E} X}}
\\
\inferrule[Unbiased-Mismatch]
  { \env \Gamma {\rejectsPrefixesOf {S_1} X} \\
    \env \Gamma {\rejectsPrefixesOf {S_2} X}
  }
  {\env \Gamma {\rejectsPrefixesOf {\gOr {S_1} {S_2}} X}}

\inferrule[Biased-Mismatch]
  { \env \Gamma {\rejectsPrefixesOf {S_1} X} \\
    \env \Gamma {\rejectsPrefixesOf {S_2} X}
  }
  {\env \Gamma {\rejectsPrefixesOf {\gOrB {S_1} {S_2}} X}}

\inferrule[Not-Front]
  {\env \Gamma {\rejectsPrefixesOf {S_1} X}}
  {\env \Gamma {\rejectsPrefixesOf {\gBind x {S_1} {S_2}} X}}

\inferrule[Not-Back]
  {\env \Gamma {\parser {S_1} v X Y} \\
   \env {\Gamma, x=v} {\rejectsPrefixesOf {S_2} Y}
  }
  {\env \Gamma {\rejectsPrefixesOf {\gBind x {S_1} {S_2}} {(X \append Y)}}}

\inferrule[Not-Nested]
  {\env \Gamma {\expr E v} \\
   \env \Gamma {\rejectsPrefixesOf P v}
  }
  {\env \Gamma {\rejectsPrefixesOf {\gParse P E} X}}

\inferrule[No-Case]
  {\env \Gamma {\expr E v} \\
   \env \Gamma {\rejectsPrefixesOf {\caseSelect v A} X}
  }
  {\env \Gamma {\rejectsPrefixesOf {\gCase E A} X}}

\end{mathpar}
\caption{
$\env \Gamma {\rejectsPrefixesOf S X}$ asserts that $X$ is not accepted by $S$
in the sense described before.
}

\end{figure}

\section{Set vs. State Transformer Semantics}

\[
(v,X,Y) \in \setSem S \iff (v,Y) \in \funSem S~(X \append Y)
\]

Using state transformers is a more powerful abstraction than what is
expressible in Core. In particular, consider an extension of Core
that allows for direct stream manipulation, $\gSetS E$, which returns
no interesting semantic value, but modifies the stream that we are parsing.
Such a construct is readily expressible using the state transformers semantics:

\[
\funSem{\gSetS E} X = \set {((),\sem E)}
\]

We cannot, however, express such a parser using the set-based semantics,
because in this formalism, parsers declare constraints on a global stream,
but they cannot change the actual stream.  One attempt to define the
semantics of such a construct could be:

\[
\setSem{\gSetS E} = \set { ((),[], \sem E) }
\]
This, however, is incorrect because instead of changing the stream,
we are making a look-ahead assertion about what the stream should be.
Thus, we'll reject any inputs that do not match $E$, which is quite
different than the intended semantics, which is a parser that never fails
but modifies the input. As a concrete example, consider $\gSetS {\eString a}$:
the input ${\eString b}$ is accepted by the transformer interpreation but
not by the (incorrect) set interpreation.



\end{document}
